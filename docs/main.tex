\documentclass{article}

\begin{document}
\section{Theory}
\subsection{Basic Components}
The following are the most basic components that are present in most genetic algorithms:
\begin{enumerate}
    \item Fitness function for optimization.
    \item Population of chromosomes.
    \item Selection of chromosomes for reproduction.
    \item Crossover to produce the following generation of chromosomes.
    \item Random mutation of chromosomes in the following generation.
\end{enumerate}
\medbreak
\subsubsection{Fitness Function}
Fitness is a term from biology which defines the extent to which a certain type of organism is able to pass itself onto the next generation, based on the way a given organism does its job. It can be stated with the following question: which survival form will pass the most copies of itself onto the next generation?\smallbreak
In the context of genetic algorithms, fitness ...

\subsubsection{Chromosomes}
Usually, a genetic problem is designed to solve a certain problem. Chromosomes can thus be described as the values an individual unit contains, that are used to solve that particular problem. For each individual unit, their corresponding chromosomes are defined in an arrary that contains parameters. The representation of the values that the parameters define is dependent on the problem that the creator is trying to solve. A chromosome can be defined as following:\smallbreak
chromosome = $\left[ p_{1} ... p_{N} \right]$\smallbreak
Where $N$ is the number of dimensions of the problem and $p$ corresponds to the given parameters. $p$ can be expressed in, for example, binary, real numbers and so forth.
\bigbreak
\subsubsection{Selection Operator}

\bigbreak

\newpage
\section{Bibliography}
Carr, J. (2014). An Introduction to Genetic Algorithms.\smallbreak
https://www.whitman.edu/Documents/Academics/Mathematics/2014/carrjk.pdf

‌
\end{document}